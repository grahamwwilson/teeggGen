%                    TEEGG USER GUIDE
%                    ----------------
%
% This is a PLAIN TeX document.
%
\magnification=\magstep1
\nopagenumbers
\parindent 0pt
\vsize=24 true cm \hsize=17 true cm
%
%\font\scriptrm=amr7                    \scriptfont0=\scriptrm
%\font\tentt=amtt10                     \scriptfont7=\tentt
%\mathchardef\ttplus="072B \mathchardef\ttmins="072D %Get a good e+ and e-
\def\eplus{e^{+}}    
\def\emins{e^{-}}
%
\def\veto{\mathop{\rm veto}\nolimits}
\def\cut{\mathop{\rm cut}\nolimits}
\def\nextline{\unskip\nobreak\hskip\parfillskip\break}
\newskip\skipi  \skipi= .4cm plus .4cm minus .2cm
\newskip\skipii \skipii=.8cm plus .8cm minus .4cm
%
\centerline{\bf TEEGG GUIDE \rm (Version 7.1) }
 
\vskip \skipii
The Monte Carlo event generator, TEEGG, simulates the process
$\eplus \emins \rightarrow \eplus \emins \gamma (\gamma),$
for low $Q^2$ configurations.
This guide describes the many parameters of TEEGG.
\vskip \skipi
Reference: D. Karlen, {\sl Nucl.\ Phys.\ }{\bf B289} (1987) 23.
 
\vskip \skipii
\centerline{\bf 1. General Parameters}
%\vskip -\skipi
%
%The general characteristics of the event sample is defined by the
%following parameters:
%$$\vbox{
\halign{
\hskip 1em \tt # \hfil & \vtop{\hsize=32em # \strut} \cr
EB     &  the beam energy in GeV. \cr
CONFIG & the event configuration to be generated. \cr }

This specifies what particles are to be in the detector acceptance, the remaining
are to be below the veto angle. 
The allowed choices are:

       \halign{\hskip 1em \it # \hfil &
               \vtop{\hsize 32em # \strut }\cr
       EGAMMA &  $e\gamma$  \cr
       GAMMA  &  single $\gamma$        \cr
       ETRON  &  single $e$   \cr }
       
  RADCOR {\hskip 1em} the level of radiative correction to be applied.
  Valid choices
  are:
       \halign{\hskip 1em \it # \hfil &
               \vtop{\hsize 32em # \strut}\cr
       NONE & no radiative correction is applied; only
              order $\alpha^3$. \cr
       SOFT & virtual and soft real photon correction to the order
              $\alpha^3$ cross section is applied. \cr
       HARD & double radiative Bhabha events are generated. \cr }
In order to generate an event sample correct to order $\alpha^4$,
{\it SOFT} and {\it HARD} event samples must be generated separately
and then
combined. %\strut \cr

%\strut
CUTOFF {\hskip 1em} defines the separation between soft and hard photons in the
          $\gamma e$ center of mass system (in GeV). This is used
          with {\tt RADCOR} = {\it SOFT} or {\it HARD}. An invalid choice
          for this quantity will be noted in the printed summary.

%\strut 
UNWGHT {\hskip 1em} specifies if unweighted events are to be generated. If
          {\tt UNWGHT} = {\tt .FALSE.}, then the event weight is given
          the variable {\tt WGHT}.
%}}$$
 
\vskip \skipii
\centerline{\bf 2. Acceptance Parameters}
\vskip \skipi
%
%The detector acceptance and veto is defined by the following parameters:
The following parameters determine the acceptance for the $ee\gamma$
final state:
$$\vbox{\halign{
\hskip 1em \tt # \hfil &#&# \hfil &\hskip 1em \vtop{\hsize=28em  # } \cr
EEMIN  &($E_{e\min}$&)&    Electron acceptance energy threshold (GeV) \cr
TEMIN  &($\theta_{e\min}$&)&     Electron acceptance angle (rad) \cr
EGMIN  &($E_{\gamma\min}$&)& Photon acceptance energy threshold (GeV) \cr
TGMIN  &($\theta_{\gamma\min}$&)&  Photon acceptance angle (rad) \cr
TEVETO &($\theta_{e\veto}$&)&    Electron veto angle (rad) \cr
TGVETO &($\theta_{\gamma\veto}$&)& Photon veto angle (rad) \cr
}}$$
The set of parameters required to define the acceptance depends on the
event configuration.
For a configuration where an electron is to be in the acceptance,
{\tt TEMIN}
and {\tt EEMIN} must be supplied and similarly if a photon is to be
in the acceptance, {\tt TGMIN} and {\tt EGMIN} must be supplied.
For the single $e$ configuration, the parameter {\tt TGVETO} is used
to define the photon veto.
{\tt TEVETO} is always used, as one electron is below this
veto angle for all configurations (for the single $\gamma$
configuration, both electrons are below this angle).
 
\eject
 
Additional parameters to define the acceptance for the
$ee\gamma\gamma$ final state are:
 
\vskip -\skipi
$$\vbox{\halign{
\hskip 1em \tt # \hfil &#&# \hfil &\hskip 1em \vtop{\hsize=28em  # } \cr
PEGMIN &($\phi_{e\gamma\min}$&)& Electron photon $\phi$ separation
                          threshold for $e\gamma$ config. (rad) \cr
EEVETO &($E_{e\veto}$&)& Electron veto energy threshold (GeV) \cr
EGVETO &($E_{\gamma\veto}$&)& Photon veto energy threshold (GeV) \cr
PHVETO &($\phi_{\ \veto}$&)& veto $\phi$ separation threshold (rad) \cr
}}$$
 
These parameters should not need to be changed from the default values.
An $e\gamma$ trial event is accepted only if an $e$ and $\gamma$ are
in the acceptance, and separated in $\phi$ by at least {\tt PEGMIN} rad.
A single $e$ trial event is rejected if a $\gamma$ is above {\tt TGVETO}
with energy greater than {\tt EGVETO} and separated
from the $e$ in $\phi$ by
more than {\tt PHVETO}.
A single $\gamma$ trial event is rejected if an $e$ is above {\tt TEVETO}
with energy greater than {\tt EEVETO}.
It is also rejected if a second $\gamma$ is above {\tt TGVETO}
with energy greater than {\tt EGVETO} and separated
from the first $\gamma$ in $\phi$ by
more than {\tt PHVETO}.
 
\vskip \skipi
\centerline{\bf 3. Maximum Weights}
\vskip \skipi
 
The program does not choose
the maximum event weights, and so they must be set by the user.
The maximum weights should be chosen to be as small as possible.
$$\vbox{\halign{
\hskip 1em \tt # \hfil & \vtop{\hsize=32em # \strut} \cr
WGHT1M & used in the rejection method for the first degree of freedom,
         $q^0_{+}$. \cr
WGHTMX & used in the trial event selection only if unweighted events are
         requested. If this is not larger than
         all weights of accepted events, then the event distribution
         will not be correct. A proper choice for this
         parameter can vary from 0.1 to 10 or more. \cr
}}$$
 
\vskip \skipi
\centerline{\bf 4. Other Parameters}
\vskip \skipi
 
Two remaining parameters specify the squared matrix elements to use.
$$\vbox{\halign{
\hskip 1em \tt # \hfil & \vtop{\hsize=32em # } \cr
MATRIX &
  defines the matrix element for the
  $\eplus \emins \rightarrow \eplus \emins \gamma$
  generation.
       \halign{\hskip 1em \it # \hfil &
               \vtop{\hsize 26em # \strut }\cr
       BK    & Berends and Kleiss,
               {\sl Nucl.\ Phys.\ }{\bf B228} (1983) 537. \cr
       BKM2  & as above but includes $t$ and $t'$ channel mass terms. \cr
       TCHAN & as calculated by using only the two
               $t$ channel diagrams. \cr
       EPA   & as derived from the equivalent photon
               approximation.  \cr} \cr
MTRXGG &
  defines the matrix element for the
  $\eplus \emins \rightarrow \eplus \emins \gamma \gamma$
  generation.
       \halign{\hskip 1em \it # \hfil &
               \vtop{\hsize 26em # \strut }\cr
       EPADC & as found from the EPA and double compton scattering. \cr
       BEEGG & Berends {\it et. al.},
               {\sl Nucl.\ Phys.\ }{\bf B264} (1986) 265. \cr
       MEEGG & Martinez and Miquel, UAB-LFAE 87-01.
               \cr
       HEEGG & Hybrid; {\it EPADC} for very low $Q^2$,
               {\it BEEGG} for moderate $Q^2$. \cr} \cr
}}$$
 
\vskip \skipi
\centerline{\bf 5. Defaults}
\vskip \skipi
The defaults for the parameters described above are listed here.
$$\vbox{\halign{
           {\tt #} \hfil & {\it #} \hfil \hskip 1em &
           {\tt #} \hfil & # & # \hfil \hskip 1em &
           {\tt #} \hfil & # & # \hfil \hskip 1em &
           {\tt #} \hfil & # \hfil \cr
 
 
EB    &{\rm=14.5 GeV}
                &EEMIN &=2.0 &GeV &PEGMIN&=$\pi/4$& rad &WGHT1M&=1.001\cr
CONFIG&=EGAMMA  &TEMIN &=0.72&rad &EEVETO&=0.0    & GeV &WGHTMX&=0.5  \cr
RADCOR&=NONE    &EGMIN &=2.0 &GeV &EGVETO&=0.0    & GeV &MATRIX&=%
                                                           {\it BKM2} \cr
CUTOFF&{\rm=0.25 GeV}
                &TGMIN &=0.72&rad &PHVETO&=$\pi/4$& rad &MTRXGG&=%
                                                          {\it EPADC} \cr
UNWGHT&={\tt.TRUE.}
                &TEVETO&=0.1 &rad &      &        &     &        &    \cr
      &         &TGVETO&=0.05&rad &      &        &     &        &    \cr
}}$$
 
\vfil
\end
